\documentclass[a4paper]{article}

\usepackage[letterpaper, portrait, margin=1in]{geometry}
\usepackage[english]{babel}
\usepackage[utf8]{inputenc}
\usepackage{amsmath}
\usepackage{amssymb}
\usepackage{amsxtra}
\usepackage{graphicx}
\usepackage[colorinlistoftodos]{todonotes}
\usepackage{verbatim}
\usepackage{forloop}
\usepackage{multirow}
\usepackage{caption}
\usepackage{float}
\usepackage[ruled]{algorithm2e}

\newlength\tindent
\setlength{\tindent}{\parindent}
\setlength{\parindent}{0pt}
\renewcommand{\indent}{\hspace*{\tindent}}

\newcommand{\ncut}{$N_{cut}(A,B)$}

\title{Normalized Cuts Notes}

\author{Matt Agee}

\date{\today}

\begin{document}

Read through: https://people.eecs.berkeley.edu/~malik/papers/SM-ncut.pdf
for this\\

Normalized cuts is a graph based algorithm for dividing an image into segments.
The nodes used in the algorithm are pixels (or superpixels) and the edges
connect them to their neighboring pixels. The weights of those edges are
determined based on some similarity metric. 

The value we attempt to minimize for normalized cuts is:
\begin{equation*}
    N_{cut}(A,B) = \frac{cut(A,B)}{assoc(A,V)} + \frac{cut(A,B)}{assoc(B,V)}
\end{equation*}

where we have 
\begin{align*}
    cut(A,B) = \sum_{u \in A, v \in B} w(u,v) &
    \indent &
    assoc(A,V) = \sum_{u \in A, t \in V} w(u,t)
\end{align*}

A and B are a subset of nodes in the full graph V w/ $A \cup B = V$. \\

Finding A and B minimizing \ncut provides a good segmentation
of the image represented by V. By repeatedly segmenting each susbequent 
segment we generate a set of regions for our image that can later be
used for object proposal. \\

To minimizing procedure can be turned into an eigenvalue problem for a 
Rayleigh quotient as follows:
\begin{equation*}
    min_x N_{cut}(x) = min_y \frac{y^T(D - W)y}{y^TDy}
\end{equation*}
where D is the degree matrix of V, W the association matrix, x a mask vector
defining A and B (with 1 values corresponding to A and -1 to B) and y
defined as:
\begin{align*}
    y = (1 + x) - b(1 - x) &
    \indent &
    k = \frac{\sum_{x_i > 0} d_i}{\sum_i d_i} &
    \indent &
    b = \frac{k}{1-k} = \frac{\sum_{x_i > 0} d_i}{\sum_{x_i < 0} d_i}
\end{align*}
where the $d_i$ are the diagonal entries of the degree matrix. In theory
y should take on discrete values, but in order to solve the problem we 
must approximate it as continuous. This causes there to be some ambiguity
in the values for x, causing us to make a judgment call as to how to 
partition the values. \\

Once we get our regions, we can go back and weight the boundaries via
their cut values normalized by boundary length. This can then be used 
as a thresholding criteria for defining a hierarchy of regions. 

\end{document}
